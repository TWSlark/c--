	\newpage
\section{Ogólne określenie wymagań}		%1
%Określenie celu pracy, co chcemy uzyskać, jakie przewidujemy wyniki













\hspace{0.60cm}Pierwszym celem projektu jest wykonanie programu listy dwukierunkowej opartej na stercie. Działanie listy ma zostać zaimplementowane w klasie oraz ma zawierać metody odpowiadające za:
\begin{itemize}
    \item Dodanie elementu na początku listy
    \item Dodanie elementu na końcu listy
    \item Dodanie elementu pod wskazany indeks
    \item Usunięcie elementu z początku listy
    \item Usunięcie elementu z końca listy
    \item Usunięcie elementu z pod wskazanego indeksu
    \item Wyświetlanie listy
    \item Wyświetalnie listy w odwrotnej kolejności
    \item Wyświetlanie nasępnego elementu
    \item Wyświetlanie poprzedniego elementu
    \item Czyszczenie całej listy
\end{itemize}
Działanie klasy i metody ma zostać przetestowane w funkcji main.

\hspace{0.60cm}Drugim celem projektu jest zapoznanie się z programem do kontroli wersji GitHub. Należy stworzyć na nim konto i zapisywać postępy nad projektem. Przy oddaniu projektu zaprezentować:
\begin{itemize}
    \item Co najmniej 5 commit'ów
    \item Co najmniej jedno cofnięcie się o dwa commit'y
    \item Usunięcie jednego commit'a
\end{itemize}

Listing kodu

\begin{lstlisting}[caption=Przykładowy kod 001, label={lst:listing-cpp}, language=C++]
#include <iostream>
#include <cstdlib>
#include <ctime>
using namespace std;

/*
liczby pseldolosowe
*/

int main(int argc, char** argv) {
	
	int tab[10][10];
	
	for(int i=0;i<10;i++)
	for(int j=0;j<10;j++)
	tab[i][j]=0;
	
	srand(time(NULL));		//generowanie z czasu
	int min=3;
	int max=7;
	for(int i=0;i<10;i++)
	for(int j=0;j<10;j++)		
	tab[i][j]=(rand()%(max-min+1))+min;	
	
	for(int i=0;i<10;i++)
	{
		for(int j=0;j<10;j++)
		cout<<tab[i][j]<<" ";	
		cout<<endl;
	}
	
	return 0;
}
\end{lstlisting}

Kod \ref{lst:listing-cpp} (s. \pageref{lst:listing-cpp}) przedstawia sposób generowania liczb pseudolosowych. Kod \ref{lst:listing-cpp2} (s. \pageref{lst:listing-cpp2}) przedstawia generowanie pliku HTML.

Alternatywna metoda wklejenia kodu:

\lstinputlisting[caption=Przykładowy kod 002, label={lst:listing-cpp2}, language=C++]{kod/main.cpp}