	\newpage
\section{Ogólne określenie wymagań}		%1
%Określenie celu pracy, co chcemy uzyskać, jakie przewidujemy wyniki













\hspace{0.60cm}Pierwszym celem projektu jest wykonanie programu listy dwukierunkowej opartej na stercie. Działanie listy ma zostać zaimplementowane w klasie oraz ma zawierać metody odpowiadające za:
\begin{itemize}
    \item Dodanie elementu na początku listy
    \item Dodanie elementu na końcu listy
    \item Dodanie elementu pod wskazany indeks
    \item Usunięcie elementu z początku listy
    \item Usunięcie elementu z końca listy
    \item Usunięcie elementu z pod wskazanego indeksu
    \item Wyświetlanie listy
    \item Wyświetalnie listy w odwrotnej kolejności
    \item Wyświetlanie nasępnego elementu
    \item Wyświetlanie poprzedniego elementu
    \item Czyszczenie całej listy
\end{itemize}
Działanie klasy i metody ma zostać przetestowane w funkcji main.

\hspace{0.60cm}Drugim celem projektu jest zapoznanie się z programem do kontroli wersji GitHub. Należy stworzyć na nim konto i zapisywać postępy nad projektem. Przy oddaniu projektu zaprezentować:
\begin{itemize}
    \item Co najmniej 5 commit'ów
    \item Co najmniej jedno cofnięcie się o dwa commit'y
    \item Usunięcie jednego commit'a
\end{itemize}